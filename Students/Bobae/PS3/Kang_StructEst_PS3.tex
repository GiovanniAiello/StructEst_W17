\documentclass[letterpaper,12pt]{article}
\usepackage{array}
\usepackage{threeparttable}
\usepackage{geometry}
\geometry{letterpaper,tmargin=1in,bmargin=1in,lmargin=1.25in,rmargin=1.25in}
\usepackage{fancyhdr,lastpage}
\pagestyle{fancy}
\lhead{}
\chead{}
\rhead{}
\lfoot{}
\cfoot{}
\rfoot{\footnotesize\textsl{Page \thepage\ of \pageref{LastPage}}}
\renewcommand\headrulewidth{0pt}
\renewcommand\footrulewidth{0pt}
\usepackage[format=hang,font=normalsize,labelfont=bf]{caption}
\usepackage{listings}
\lstset{frame=single,
  language=Python,
  showstringspaces=false,
  columns=flexible,
  basicstyle={\small\ttfamily},
  numbers=none,
  breaklines=true,
  breakatwhitespace=true
  tabsize=3
}
\usepackage{amsmath}
\usepackage{amssymb}
\usepackage{amsthm}
\usepackage{harvard}
\usepackage{setspace}
\usepackage{float,color}
\usepackage[pdftex]{graphicx}
\usepackage{hyperref}
\hypersetup{colorlinks,linkcolor=red,urlcolor=blue}
\theoremstyle{definition}
\newtheorem{theorem}{Theorem}
\newtheorem{acknowledgement}[theorem]{Acknowledgement}
\newtheorem{algorithm}[theorem]{Algorithm}
\newtheorem{axiom}[theorem]{Axiom}
\newtheorem{case}[theorem]{Case}
\newtheorem{claim}[theorem]{Claim}
\newtheorem{conclusion}[theorem]{Conclusion}
\newtheorem{condition}[theorem]{Condition}
\newtheorem{conjecture}[theorem]{Conjecture}
\newtheorem{corollary}[theorem]{Corollary}
\newtheorem{criterion}[theorem]{Criterion}
\newtheorem{definition}[theorem]{Definition}
\newtheorem{derivation}{Derivation} % Number derivations on their own
\newtheorem{example}[theorem]{Example}
\newtheorem{exercise}[theorem]{Exercise}
\newtheorem{lemma}[theorem]{Lemma}
\newtheorem{notation}[theorem]{Notation}
\newtheorem{problem}[theorem]{Problem}
\newtheorem{proposition}{Proposition} % Number propositions on their own
\newtheorem{remark}[theorem]{Remark}
\newtheorem{solution}[theorem]{Solution}
\newtheorem{summary}[theorem]{Summary}
%\numberwithin{equation}{section}
\bibliographystyle{aer}
\newcommand\ve{\varepsilon}
\newcommand\boldline{\arrayrulewidth{1pt}\hline}


\begin{document}

\begin{flushleft}
  \textbf{\large{Problem Set \#3}} \\
  MACS 40200, Dr. Evans \\
  Bobae Kang
\end{flushleft}

\vspace{5mm}

\begin{enumerate}
\item\textbf{ MLE estimation of simple macroeconomic model.}
\begin{enumerate}
\item Use the data $(w_t, k_t)$ and equations (3) and (5) to estimate the four parameters $(\alpha, \rho, \mu, \sigma)$ by maximum likelihood. Report your estimates and the inverse
hessian variance-covariance matrix of your estimates.
\par\bigskip
MLE estimate for $\alpha$ = 0.457477074054 \par
MLE estimate for $\rho$ = 0.720516972669 \par
MLE estimate for $\mu$ = 10.1865571199 \par
MLE estimate for $\sigma$ = 0.0919962218857 \par
Log-likelihood value = -96.7069080209\par
VCV = [[ 1.  0.  0.  0.  0.] \par
\hspace{12mm}   [ 0.  1.  0.  0.  0.] \par
\hspace{12mm}   [ 0.  0.  1.  0.  0.] \par
\hspace{12mm}   [ 0.  0.  0.  1.  0.] \par
\hspace{12mm}   [ 0.  0.  0.  0.  1.]]
\par\bigskip

\item Now we will estimate the parameters another way. Use the data $(w_t, k_t)$ and equations (4) and (5) to estimate the four parameters $(\alpha, \rho, \mu, \sigma)$ by maximum likelihood. Report your estimates and the inverse hessian variance-covariance
matrix of your estimates.
\par\bigskip
MLE estimate for $\alpha$ = 1e-10 \par
MLE estimate for $\rho$ = 0.25074887437 \par
MLE estimate for $\mu$ = 0.30025181993 \par
MLE estimate for $\sigma$ = 0.096985710088 \par
Log-likelihood value = -91.4253132108\par
VCV = [[  1.           8.25029459  -5.62198726  -0.72803545] \par
\hspace{12mm}   [  8.25029459  70.01646074 -44.06310322  -5.88712085] \par
\hspace{12mm}   [ -5.62198726 -44.06310322  34.37234784   4.22175443] \par
\hspace{12mm}   [ -0.72803545  -5.88712085   4.22175443   0.58265255]] \par
\par\bigskip

\item According to your estimates from part (a), if investment/savings in the current period is $k_t$ = 7,500,000 and the productivity shock in the previous period was $z_{t-1}$ = 10, what is the probability that the interest rate this period will be greater than $r_t$ = 1.
\par\bigskip
$Pr(r_t > 1 | \hat{\theta}, k_t = 7,500,000, z_{t-1} = 10) =  1.$
\end{enumerate}
\end{enumerate}


\end{document}
